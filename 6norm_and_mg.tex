\chapter{基于时序逻辑的表示与决策方法}
本章提出一种基于LTL的智能体目标和norm的表示方法,并将其与\SA 算法结合,提出一种同时支持实现型目标、维持型目标以及norm的意图调度算法\SAT。本章实验部分对SAT的性能表现在不同的资源储备量下进行了分析,并与Duff等人提出的PMG\cite{DBLP:conf/atal/DuffHT06}算法和Meneguzzi等人提出的v-BDI算法进行了比较,实验结果表明\SAT 的性能表现相对PMG和v-BDI有显著优势\footnote{在前两章中,本文分别将SAM与PMG进行对比,将SAN与v-BDI进行对比。}。
\section{多类型的目标与norm}
在许多社会仿真场景中,目标可能有丰富的类型,例如需要智能体在一段时间内保持某一特定状态的维持型目标,或是在某些特定情况下,只有在满足某些特定条件时才最求的条件目标。此外,智能体的行为可能受到norm的约束,智能体需要在决策时加入对norm的考虑,以避免违反norm,或者选择性地违反norm以实现更重要的目标。本文第\ref{mg}章研究了了如何同时对实现型目标和维持型目标进行意图调度,并提出了SAM算法;第\ref{norm}章研究了了如何在norm约束下进行意图调度,并提出了SAN算法。本章考虑更为实际的社会仿真模拟场景:在本章中,各种类型的目标与norm统一以时序逻辑(Linear Temporal Logic, LTL)的形式表示,而智能体需要对LTL表示的目标或norm进行意图调度。

Dastani等人\cite{}提出了一种基于LTL表示各种类型目标(如实现型目标、维持型目标、行动目标等)的框架。在其框架中一个维持型目标被表示为$\alpha \cup \tau$,其中$\alpha$为智能体需要维持的条件,$\cup$表示“直到”,而$tau$表示终止条件。因此$\alpha \cup \tau$表示维持$\alpha$,直到$\tau$被满足。LTL的表示能力是通用且强大的,足以表示各种类型的目标并拓展现有目标类型。为了实现这些LTL形式的目标Dastani等人\cite{}也在其研究中提出一种转换方法,能够将LTL公式转变为一般智能体编程语言所支持的普通实现型目标和维持型目标。其LTL目标的转化规则以条件-动作对(Condition-Action Pairs)来表示。其描述了在某个特定环境状态下的目标状态转变(由LTL目标转化为基本目标)。Dastani等人\cite{}描述了两种类型的条件-动作对:通用性的以及领域专用的条件-动作对。通用性的条件动作对适用于该框架给定的目标形式,而领域专用的条件-动作对则是为特定领域设计的针对特殊表示形式目标的转换方法,目的是提升智能体的适用范围。然而,该框架并没有提供一个一般性的意图调度算法,某些特殊形式的目标仍然需要用户自定义条件-动作对,虽然有一点的拓展性,但加重了用户的负担;此外,该框架也没有考虑到社会仿真场景下的norm,这导致其不适用于norm约束场景。

Gutierrez等人提出了用于向智能体发出指令的语义模型。该研究以LTL公式的形式表示指令,而智能体则必须以确保LTL公式得到满足的方式行动。此外,该研究假定智能体有一定的后台安全需求(同样以LTL形式表示)需要智能体保护(类似于维持型目标)。该研究对LTL满足(如何行动以满足LTL)与LTL生成(如何根据需求生成LTL)的时间与空间复杂度进行了细致分析。然而,正如其作者所提到,将该语义模型应用于实际的好处仍然未知,因为其并没有被实际应用,无相关实验分析,而仅为一种可能有价值的理论。

Krzisch等人\cite{}基于经典的规划问题对norm系统进行建模。其考虑了两种norm形式化的方法:一种是仅仅考虑某些场景下的动作(允许或是禁止执行某些动作);另一种则是LTL约束下的一系列状态转移。该研究提出了相应算法以应对norm的约束,然而其算法并没有考虑到多种类型的目标,并且也忽略了多个并发意图下的意图进展问题。

Paxton等人\cite{}提出了一种基于深度强化学习的方法以解决复杂动态环境中的规划问题。而该问题被规范表示为一组LTL公式。该方法将MCTS与基于LTL规范训练的人工神经网络结合。该研究在一个仿真自动驾驶场景对人工神经网络的学习能力进行了评估。然而,其在模拟场景下的学习效果并不如预期,并且智能体偶尔会出现奇怪行为。另外,该研究的重点也与对智能体目标、norm的表示关联不大。


\section{使用LTL表示目标与norm}
使用LTL表示目标和norm的优势在于智能体可以在更高的抽象层次处理目标和norm的特性,并且更意图拓展目标和norm的类型。
\section{基于目标计划树模型的问题定义}
\section{\SAT 意图调度算法}
\section{实验}
\subsection{低电池容量实验}
\subsection{高电池容量实验}
\section{本章总结}
