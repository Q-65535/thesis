\chapter{基础知识与问题定义}
本章首先介绍BDI智能体信念、动作计划和目标的规范表示,然后介绍一种表示BDI智能体意图的数据结构-目标计划树(Goal-Plan Tree,GPT),最后基于此对意图进展问题进行规范定义。
\section{信念、计划以及动作的规范表示}
本节主要介绍BDI智能体信念、动作、计划以及目标这四个基本元素的规范表示,在后续的章节中根据问题场景不同将会介绍到BDI智能体其他属性的规范表示。
\paragraph{信念}
信念为智能体对环境及其自身的认知,一个信念以一个实例化的一阶谓词公式表示:
$$P(t_1, \ldots,t_n)$$
其中$P$为谓词,$t_1, \ldots,t_n$为实例化的个体变元。例如有一谓词公式$Battery(\beta, x)$表示智能体$\beta$当前的电量为$x$,其中$\beta$为一变量指代某一个智能体,$x$为一常数变量,则$Battery(Alice, 60)$这条信念表示智能体$Alice$当前的电量为$60$。
\paragraph{动作}
动作是智能体可以执行的最小单元,其可以直接对环境造成影响。一个动作以一个三元组表示:
$$<A,C_{pre},C_{post}>$$
其中$A$为该动作的名称,$C_{pre}$和$C_{post}$分别对应该动作的前置条件和后置条件。与信念的表示相同,每个条件(前置条件和后置条件)以一阶谓词公式表示。
\paragraph{计划}
每个计划P以$G:\emptyset \gets \alpha_1; \dots;\alpha_n$的形式表示,其中G为一个目标,$\emptyset$是一组命题公式,为该计划的前置条件,只有当$\emptyset$为真时,P才能被执行。最后,$\alpha_1; \dots;\alpha_n$为计划中的一些列执行步骤,计划的执行对于依次执行这些执行步骤。每个执行步骤或为可以直接改变环境的动作或为子目标,而子目标可其对应的子计划实现。
\paragraph{目标}
$$<G_a,C_g,Pls>$$
其中$G_a$为该目标的名称,$C_g$为目标条件,即一组智能体想要达到的条件。每个目标都与一组用于实现该目标的计划$Pls=\{P_1,\dots,P_n\}$相关联。一旦目标条件被达成,智能体便会抛弃该实现型目标。

\section{目标计划树}
% ref
Thangarajah等人\cite{DBLP:journals/jar/ThangarajahP11,DBLP:conf/ijcai/ThangarajahPW03,DBLP:conf/ijcai/ThangarajahPW03,DBLP:conf/ecai/ThangarajahWPF02}提出目标计划树(Goal-Plan Tree)的数据结构来表示BDI智能体目标及计划之间的关系。在此基础上,Yao等人\cite{DBLP:conf/atal/YaoSL16}提出了对目标计划树的拓展,增加了对动作的表示,并定义了包含多种可能后置条件的易错动作和有执行时间的持续型动作。目标的定义也被拓展为含有截止时间和期望完成时间。考虑到目标计划树的表达能力,本文将使用Yao等人提出的拓展目标计划树来表示智能体的意图。

% GPT
目标计划树的根节点为顶层目标节点,即智能体想要达到的最终目的;其孩子节点为一个或多个计划节点。每个计划节点的孩子节点为一系列执行步骤(动作或子目标);而子目标又有其对应的计划节点,如此便构成了树形结构,其展现了所有实现顶层目标的途径。此外,为了实现某个目标,智能体只需要选择一个计划执行即可,因此计划节点也被视为“OR”节点;而为了完整执行一个计划,智能体需要依次成功执行其对应的所有执行步骤,因此计划节点的孩子节点被视为“AND”节点。

图\ref{fig:gpt}展示了火星探测器领域下的一个简单目标计划树模型,其中环境为$h*w$的网格。火星探测器智能体可以通过执行动作在不同的方格中移动。其顶层目标$G0$表示将智能体移动到$(a,b)$位置。有五个不同的计划可用于实现$G0$,不同的计划可应用于不同的环境状态下。如果智能体当前位置位于$(x,y)$且$y < b$,则计划$P0$可用于实现$G0$。P0由两个步骤组成,一个是可直接执行的动作$A0$,执行$A0$可以使智能体向上移动一个单位,第二个是子目标$G0$,$G0$为递归目标(即该子目标实际与顶层目标相同)。$A0$的前置条件为智能体不在网格的上边界(在上边界执行该动作会使得智能体超出边界范围),而其后置条件则为将智能体位置至于$(x,y+1)$。另外,该目标计划树示例将用于后续章节的实验部分。
\begin{figure*}[htb]
\centering
\includegraphics[scale=0.23]{./figs/MarsRover_GPT}
\bicaption{目标计划树例子}{Example goal-plan tree}
\label{fig:gpt}
\end{figure*}

% Benefits of GPT representation
目标计划树直观地展现了智能体目标、计划和动作之间的关系。除此以外,目标计划树还可用于记录实现目标或执行计划和动作所需的前置条件(Precondition)以及执行的结果,后置条件(Postcondition)。计划的前置条件指的是执行计划前需要满足的条件,如果前置条件不满足,则计划无法执行(动作的前置条件同理)。后置条件指的是执行动作之后的结果,对智能体所处环境造成的影响。最后,目标计划树还可以用于刻画智能体在某个环境下实现特定目标时的健壮性,Thangarajah等人基于目标计划树提出Coverage和Overlap的概念\cite{DBLP:conf/aamas/ThangarajahSP12}以刻画智能体程序的健壮性。其中Coverage表示一个意图当前的目标或子目标至少有一个计划可执行情况下可能的环境状态占比;而Overlap则表示一个目标有多个(大于一个)可执行计划的情况下可能的环境状态占比。

\section{意图进展问题}
上一小节介绍了目标计划树的概念,本章节基于目标计划树,对BDI智能体意图进展问题进行规范定义。

BDI智能体通常有多个计划用以实现某一个目标,不同的计划根据其前置条件不同适用于不同的环境场景,前置条件满足的计划被称为可应用计划(Applicable Plan)。
% plan selection
当同时有多个可应用计划是,智能体需要考虑选择哪一个计划用以实现相应目标,该问题被称为计划选择问题(Plan Selection Problem)。计划选择会影响到并发执行的其他目标,因为不同的计划有着不同的后置条件,这些后置条件可能会影响其他目标下计划或动作的前置条件。这种影响可能是负面的,例如一个计划执行之后使得其他目标下的某些计划的前置条件失效,也有可能是正面的,例如一个目标下的某些计划的前置条件在初始时是不满足的,当执行其他目标下的计划之后使得其前置条件满足。

% intention selection
另一方面,当智能体有多个意图时,下一步该执行哪个意图的问题被称为意图选择问题(Intention Selction Problem)。意图选择问题同样也会影响目标的实现,不合理的意图选择可能会造成各个意图中的前置条件相互破坏,导致无法实现任何一个目标(单个目标本身可以顺利实现),而合理的意图选择则可以使得多个意图产生协同效应,相互促进执行。

% IPP
意图进展问题(Intention Progression Problem)为上述两问题的结合,即同时考虑计划选择问题和意图选择问题。意图进展问题是本文所研究的问题的基础。为了规范并一般化地表示意图进展问题,本文基于目标计划树对意图进展问题进行定义。一个意图的进展实际对应于一个目标计划树中一条从根节点到一个叶子结点的路径。目标计划树中的任何一条从根节点至叶子节点的路径都对应于实现其顶层目标的一种方式。每一条路径都包含一系列的计划、动作和子目标,如果他们都被成功执行,即可实现顶层目标。如此,智能体的执行过程可看作多个目标计划树执行路径的相互交错,即对应于意图进展的并发执行。

%
在不同场景下对智能体性能的评估标准可能不同,例如,有时智能体完成目标的数量越多越好,而有时公平性更重要,多个目标的完成时间越接近越好。为了在社会仿真模拟场景下提供一个一般化的意图调度方法,本文假设对某一场景下的智能体有一价值函数$f_{\mu}$,$f_{\mu}$的输入参数为智能体状态信息,并返回一个评估值,代表智能体当前的性能表现。$f_{\mu}$函数可由用户自定义,以应用于不同的问题场景。

% definition
基于上述内容,意图进展问题可被理解为:如何选择多个目标计划树的交错执行路径,使得最终基于$f_{\mu}$获得最大的价值。具体的,给定一组表示智能体意图的目标计划树集合$\{t_1, \dots, t_n\}$,当前环境信息$Env$,智能体的初始状态$S_0$,需要实现的目标集合$G_0$以及当前问题中的价值函数$f_{\mu}$,智能体在每一个运行周期中求解并执行目标选择,计划选择以及意图选择,直到某个最终状态$S$停止运行,要求最终状态$S$使得价值函数$f_{\mu}$的值最大,即智能体从初始状态$s_0$开始进行目标计划树的交错执行,不存在某个最终状态$S^{\prime}$,使得$f_{\mu}(S^{\prime})$的值大于$f_{\mu}(S)$。

% \subsection{对意图进展问题的拓展}
考虑到智能体应用场景的多样性,本文对上述意图进展问题进行拓展作为本文的主要研究问题。本文考虑三类对意图进展问题的拓展:维持型目标下的意图进展问题,norm约束下的意图进展问题以及以LTL为输入的意图进展问题。

\paragraph{维持型目标下的意图进展问题}
考虑到维持型目标通常应用在资源有限的场景中,在该问题定义中,价值函数考虑到了资源消耗。具体地,基于目标计划树模型,维持型目标下的意图进展问题可被定义为:给定一组表示智能体意图的目标计划树$\{t_1, \dots, t_n\}$以及一组表示智能体当前环境状态的条件变量$Env$,在每一个执行周期中返回一个目标计划树$t_i$中的下一个执行步骤使得智能体实现目标所获得的收益最多且消耗的资源量最少。
\paragraph{norm约束下的意图进展问题}
该问题定义对意图进展问题的原始定义进行扩展,加入norm的限制,使得智能体必须同时考虑目标(如何实现目标)以及norm(如何避免违反norm)。具体地,基于目标计划树模型,维持型目标下的意图进展问题可被定义为:给定一组表示智能体意图的目标计划树$\{t_1, \dots, t_n\}$、一组表示当前norm的集合 $N$。以及一组表示智能体当前环境状态的条件变量$Env$,在每一个执行周期中返回一个目标计划树$t_i$中的下一个执行步骤使得智能体实现目标所获得的收益最多且违反norm受到的惩罚最少。
\paragraph{以LTL对象为输入的意图进展问题}
该问题定义加入对LTL表示的目标与norm的考虑。具体地,基于目标计划树模型,以LTL为输入的意图进展问题可被定义为:给定一组表示智能体意图的目标计划树$\{t_1, \dots, t_n\}$,一组智能体需要满足的LTL公式与其对应价值以及一组表示智能体当前环境状态的条件变量$Env$,在每一个执行周期中返回一个目标计划树$t_i$中的下一个执行步骤使得智能体获得的总价值最高。
