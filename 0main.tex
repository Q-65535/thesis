\begin{abstract}
智能体(Agent)是嵌于某特定环境下能够自主行动以实现给定目标的个体。多智能体系统(Multi-Agent System, MAS)通过智能体间的交互实现分布式智能,能够用于解决单个智能体难以或无法解决的问题,是当前人工智能领域的研究热点之一。
%
在当前对智能体的研究中出现多种智能体体系结构和模型,其中Belief-Desire-Intention-based (BDI)智能体模型是当前学术界应用和研究最多的智能体体系结构之一。
%
信念-愿望-意图智能体的关键优势之一是它能够同时追求多个目标。基于Summary-Infomation(SI)的方法、基于Coverage的方法和基于蒙特卡洛树搜索(Monte-Carlo Tree Search)的方法被提出来调度智能体的意图进程以实现并发目标。然而,所有这些现有的工作都局限于调度实现型目标。在实际情况中,智能体可能被要求在一段时间内保持环境状态,而不是实现某些状态。此外,大多数现有的方法也没有考虑到社会准则,如规范norm。他们对智能体意图进行调度的主要目的是为了最大化实现目标的数量,而完全不考虑norm约束。

% 关于本文内容
本文研究了当存在维持型目标和norm约束时调度智能体意图的技术。具体地,本文基于当前基于蒙特卡洛树搜索的调度器\SA,提出一种新的智能体意图调度器\SAM。\SAM 一方面继承\SA 的优势,即高效利用不同意图之间的协同效应,有效避免意图间的相互冲突,另一方面可以在决策时考虑维持型目标或norm以更好地以用户自定义的方式实现目标。本文在火星探测器模拟场景中对\SAM 的性能进行评估,并表明本文的方法在不同困难的一系列场景中优于其他最先进的方法。
\end{abstract}